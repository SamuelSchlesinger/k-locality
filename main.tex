\documentclass{article}
\usepackage{amsmath}
\usepackage{amssymb}
\usepackage{amsthm}
\usepackage{biblatex}

\addbibresource{bibliography.bib}

\newcommand{\LC}{\mathcal{L}}

\newtheorem{definition}{Definition}
\newtheorem{theorem}{Theorem}
\newtheorem{lemma}{Lemma}
\newtheorem{corollary}{Corollary}
\newtheorem{remark}{Remark}

\title{Localization Complexity: Core Definitions and Circuit Connections}
\author{Samuel Schlesinger, Joshua Grochow}
\date{\today}

\begin{document}

\maketitle

\begin{abstract}
We define the $k$-localization complexity $\LC_k(D)$ of a distribution $D$ as the minimum number of latent variables needed to realize $D$ as the observed marginal of a $k$-local maximum-entropy model. We then isolate a single technical mechanism---\emph{local verification}---that converts bounded fan-in circuit computations into $k$-local models. This yields three clean upper bounds:
\[
\LC_k(D) \leq G_{k-1}(D),\qquad
\LC_k(D) \leq C_{k-1}(S) \text{ for flat } D,\qquad
\LC_k(D) \leq W_{k-1}(D).
\]
The paper is intentionally restricted to the foundational definitions, core lemmas, and proofs of these connections.
\end{abstract}

\section{Introduction}
Complexity measures for distributions ask how succinctly a computational model can represent a target distribution \cite{vi10}. We introduce and study one such measure based on local maximum-entropy structure.

The purpose of this paper is narrow: define the model precisely and prove its direct connections to bounded fan-in circuits. We do not pursue computational hardness questions here.
% TODO(rebuild): Re-add a full "Preliminaries" section if/when we expand scope,
% including entropy/exponential-family background and any required circuit conventions.

\paragraph{Contributions.}
\begin{itemize}
\item We define $k$-local distributions, $k$-localizations, and localization complexity $\LC_k(D)$.
\item We prove a general local-verification theorem: if local constraints forbid every invalid global configuration, then the uniform distribution on valid traces is a maximum-entropy feasible point.
\item Using this theorem, we prove upper bounds from three circuit-style measures: generator complexity, flat-support circuit complexity, and witness-counting complexity.
\end{itemize}

\section{Model and Basic Definitions}
Fix binary random variables $X_1,\dots,X_n$ and a distribution $D$ with mass function $\rho:\{0,1\}^n\to[0,1]$. Let
\[
S=\{x\in\{0,1\}^n:\rho(x)>0\}
\]
be the support.

\begin{definition}[Marginal model]
A distribution $D'$ over variables $X_1,\dots,X_n,H_1,\dots,H_m$ is a marginal model of $D$ if
\[
\forall x\in\{0,1\}^n,\qquad
\sum_{h\in\{0,1\}^m}\Pr_{D'}(X=x,H=h)=\rho(x).
\]
\end{definition}

\begin{definition}[$k$-local distribution]
A distribution $Q$ over $N$ binary variables is \emph{$k$-local} if there exists a finite family of subsets $S_1,\dots,S_t\subseteq[N]$ with $|S_i|\le k$ and target marginals $\phi_i:\{0,1\}^{S_i}\to[0,1]$ such that:
\begin{itemize}
\item $Q$ matches every constraint $\phi_i$ on $S_i$, and
\item among all distributions matching these constraints, $Q$ has maximum Shannon entropy.
\end{itemize}
\end{definition}

\begin{definition}[$k$-localization complexity]
A marginal model $D'$ of $D$ is a \emph{$k$-localization} of $D$ if $D'$ is $k$-local. For $k\ge2$,
\[
\LC_k(D)=\min\{m:\text{$D$ has a $k$-localization with $m$ latent variables}\}.
\]
\end{definition}

\begin{lemma}[Monotonicity]
If $k_1\ge k_2$, then $\LC_{k_1}(D)\le \LC_{k_2}(D)$.
\end{lemma}

\begin{proof}
Any constraint on at most $k_2$ variables is also a constraint on at most $k_1$ variables.
\end{proof}

\section{A Local Verification Theorem}
The following lemma is the core mechanism used in all circuit reductions.
% TODO(rebuild): Consider re-introducing a support-intersection style lemma with
% precise regularity assumptions (or a corrected alternative) if needed for converse bounds.

\begin{theorem}[Local verification]\label{thm:local-verification}
Let $T\subseteq\{0,1\}^N$ be nonempty, and let $U_T$ be the uniform distribution on $T$. Let $\{(S_i,\phi_i)\}_{i=1}^t$ be local constraints where each $|S_i|\le k$ and each $\phi_i$ is the $S_i$-marginal of $U_T$.

Assume the following \emph{local completeness} property:
\[
\forall y\notin T,\ \exists i\in[t]\ \text{s.t.}\ \phi_i(y_{S_i})=0.
\]
Then $U_T$ is a maximum-entropy distribution among all distributions satisfying these constraints. In particular, $U_T$ is $k$-local.
\end{theorem}

\begin{proof}
$U_T$ is feasible by construction.

Let $Q$ be any feasible distribution. If $Q(y)>0$, then for every $i$, the tuple $y_{S_i}$ has positive $Q$-marginal. Since $Q$ matches $\phi_i$, this implies $\phi_i(y_{S_i})>0$. By local completeness, this is impossible when $y\notin T$. Hence $\mathrm{supp}(Q)\subseteq T$.

Therefore
\[
H(Q)\le \log |\mathrm{supp}(Q)|\le \log|T|=H(U_T),
\]
where equality on the right is the entropy of the uniform distribution on $T$. Since $U_T$ is feasible, it is a maximum-entropy feasible point.
\end{proof}

\section{Circuit-Based Upper Bounds}
We now instantiate Theorem~\ref{thm:local-verification} for three circuit models.

\subsection{Generator Complexity}

\begin{definition}[Generator complexity]
For $r\ge1$, a circuit $C$ with fan-in at most $r$ \emph{generates} $D$ if, on uniform random input $u\in\{0,1\}^{m(C)}$, its $n$ output bits have distribution $D$. Define
\[
G_r(D)=\min_C \big(m(C)+s(C)\big),
\]
where $s(C)$ is the number of non-output internal gates.
We use the standard acyclic deterministic circuit model, and each output bit is the value of a designated output gate (not counted in $s(C)$).
\end{definition}

\begin{theorem}\label{thm:generator}
For every $k\ge2$,
\[
\LC_k(D)\le G_{k-1}(D).
\]
\end{theorem}

\begin{proof}
Take a fan-in-$(k-1)$ generator circuit $C$ for $D$ with $m$ random input bits and $s$ non-output gates. For each input $u\in\{0,1\}^m$, let $g(u)$ be the full vector of internal gate values and let $x(u)$ be the output vector. Define
\[
T=\{(u,g(u),x(u)):u\in\{0,1\}^m\}.
\]
Let $U_T$ be uniform on $T$, with observed variables $X:=x(u)$ and latent variables $(u,g(u))$.

For each computed node (internal gate or output gate), add one local constraint on that node together with its at most $k-1$ parent nodes, using the corresponding marginal of $U_T$. Each such constraint has arity at most $k$.

If a global assignment satisfies all these local truth-table relations, then by topological induction over the DAG every computed node value is forced to be exactly the value computed from $u$, so the assignment lies in $T$. Contrapositively, if $y\notin T$, some computed node violates its local relation. For that node, the corresponding local tuple has probability $0$ under $U_T$, so local completeness holds. By Theorem~\ref{thm:local-verification}, $U_T$ is $k$-local.

Finally, the observed marginal equals $D$:
\[
\Pr_{U_T}(X=x)=\frac{|\{u:C(u)=x\}|}{2^m}=\rho(x).
\]
So $U_T$ is a $k$-localization with $m+s$ latent variables, proving $\LC_k(D)\le m+s$. Minimizing over $C$ gives the claim.
\end{proof}

\subsection{Flat Distributions and Recognizer Circuits}

\begin{definition}
Let $C_r(S)$ be the minimum number of gates (including the output gate) in a fan-in-$r$ circuit that recognizes $S\subseteq\{0,1\}^n$.
\end{definition}

\begin{theorem}\label{thm:flat}
If $D$ is flat on support $S$, then for every $k\ge2$,
\[
\LC_k(D)\le C_{k-1}(S).
\]
\end{theorem}

\begin{proof}
Let $C$ be a fan-in-$(k-1)$ recognizer for $S$ with $s$ gates total and output bit $o(x)\in\{0,1\}$. For each $x\in S$, let $g(x)$ be the full gate-value vector (including $o(x)$) and define
\[
T=\{(x,g(x)):x\in S\}.
\]
Let $U_T$ be uniform on $T$, with observed variables $x$ and latent variables $g(x)$.

As in Theorem~\ref{thm:generator}, include local constraints for every computed-node relation. Also include the unary constraint $\Pr(o=1)=1$ (equivalently, the one-variable marginal of $o$ under $U_T$).

By the same topological-induction argument as in Theorem~\ref{thm:generator}, any assignment that satisfies all gate relations is exactly a valid trace for its visible input $x$. Therefore any assignment not in $T$ either violates a local gate relation or has $o=0$; in both cases some constrained local pattern has zero probability under $U_T$. Thus local completeness holds, so Theorem~\ref{thm:local-verification} implies $U_T$ is $k$-local.

Its observed marginal on $x$ is uniform on $S$, i.e. exactly $D$. Therefore $\LC_k(D)\le s$, and minimizing over $C$ yields the bound.
\end{proof}

\subsection{Witness-Counting Complexity}

\begin{definition}[Witness-counting model]
A fan-in-$r$ nondeterministic circuit $C(x,h)$ with $n$ visible bits $x$ and $m$ witness bits $h$ is a witness-counting model for $D$ if
\[
\sum_{x'\in\{0,1\}^n}|\{h:C(x',h)=1\}|>0
\]
and
\[
\rho(x)=\frac{|\{h:C(x,h)=1\}|}{\sum_{x'\in\{0,1\}^n}|\{h:C(x',h)=1\}|}
\quad\text{for all }x.
\]
Define
\[
W_r(D)=\min_C\big(m(C)+s(C)\big),
\]
where $s(C)$ is the total number of gates (including the output gate).
\end{definition}

\begin{theorem}\label{thm:witness}
For every $k\ge2$,
\[
\LC_k(D)\le W_{k-1}(D).
\]
\end{theorem}

\begin{proof}
Take a fan-in-$(k-1)$ witness-counting circuit $C(x,h)$ with $m$ witness bits and $s$ gates total. Let $g(x,h)$ be its full gate-value vector (including the output bit) and define
\[
T=\{(x,h,g(x,h)):\ C(x,h)=1\}.
\]
Let $U_T$ be uniform on $T$, with observed variables $x$ and latent variables $(h,g)$.

Add local constraints for each computed-node relation and the unary constraint that the output bit is $1$. As in Theorem~\ref{thm:generator}, satisfying all computed-node relations forces a valid circuit trace for $(x,h)$; with the unary output-$1$ constraint this is equivalent to $C(x,h)=1$, i.e. membership in $T$. Hence any assignment outside $T$ violates one of these local constraints, so local completeness holds. By Theorem~\ref{thm:local-verification}, $U_T$ is $k$-local.

The observed marginal is
\[
\Pr_{U_T}(X=x)=\frac{|\{h:C(x,h)=1\}|}{|T|}=\rho(x),
\]
by definition of witness-counting model. Hence this is a $k$-localization with $m+s$ latent variables. Minimizing proves the theorem.
\end{proof}

\section{Summary}
The paper isolates one reusable proof pattern: local circuit consistency constraints can force global trace validity, and once this local completeness condition is met, uniform trace distributions are maximum-entropy $k$-local models. This gives direct, clean upper bounds relating localization complexity to three bounded fan-in circuit measures.
% TODO(rebuild): Re-add lower bounds connecting \LC_k to nondeterministic complexity.
% TODO(rebuild): Re-add a section on computing \LC_k (algorithms/decision problems).
% TODO(rebuild): Re-add conditional hardness discussion (natural-proofs framing),
% but only with assumptions and statements aligned to the standard literature.
% TODO(rebuild): Re-add expanded open problems once the technical scope is restored.

\medskip
\printbibliography

\end{document}
